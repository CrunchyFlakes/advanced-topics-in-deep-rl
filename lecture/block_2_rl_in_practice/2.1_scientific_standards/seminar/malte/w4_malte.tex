\documentclass[12pt]{beamer}

\usepackage[utf8]{inputenc}
\usepackage{amsmath}
\usepackage{hyperref}
\usepackage[style=ext-authoryear, backend=biber]{biblatex}
\addbibresource{refs.bib}

\DeclareOuterCiteDelims{parencite}{\bibopenbracket}{\bibclosebracket}

\begin{document}
\begin{frame}{Formalization of replicability}
  \begin{itemize}
    \item reproducibility under multiple seeds
    \item implies good generalization and stability \parencite{reproducibility_learning}
    \item \parencite{replicability} is about tabular infinite-horizon algorithms
    \item complexity dependent on cardinality of state-action space
    \item replicability usually sample inefficient \parencite{replicability}
    \item[$\rightarrow$] very strict: $O(n^3)$ sample complexity, $O(\exp(n))$ time complexity
    \item approximate replicability: high probability of producing same policy with different seeds
    \item[$\rightarrow$] $O(n)$ sample and time complexity \parencite{replicability}
  \end{itemize}
\end{frame}

\end{document}
